% Generated by Sphinx.
\def\sphinxdocclass{report}
\documentclass[letterpaper,10pt,english]{sphinxmanual}
\usepackage[utf8]{inputenc}
\DeclareUnicodeCharacter{00A0}{\nobreakspace}
\usepackage{cmap}
\usepackage[T1]{fontenc}
\usepackage{babel}
\usepackage{times}
\usepackage[Bjarne]{fncychap}
\usepackage{longtable}
\usepackage{sphinx}
\usepackage{multirow}
\usepackage{eqparbox}


\addto\captionsenglish{\renewcommand{\figurename}{Fig. }}
\addto\captionsenglish{\renewcommand{\tablename}{Table }}
\SetupFloatingEnvironment{literal-block}{name=Listing }



\title{Resume / CV}
\date{May 25, 2018}
\release{1.0.0}
\author{David Pierson Bradway}
\newcommand{\sphinxlogo}{}
\renewcommand{\releasename}{Release}
\setcounter{tocdepth}{2}
\makeindex

\makeatletter
\def\PYG@reset{\let\PYG@it=\relax \let\PYG@bf=\relax%
    \let\PYG@ul=\relax \let\PYG@tc=\relax%
    \let\PYG@bc=\relax \let\PYG@ff=\relax}
\def\PYG@tok#1{\csname PYG@tok@#1\endcsname}
\def\PYG@toks#1+{\ifx\relax#1\empty\else%
    \PYG@tok{#1}\expandafter\PYG@toks\fi}
\def\PYG@do#1{\PYG@bc{\PYG@tc{\PYG@ul{%
    \PYG@it{\PYG@bf{\PYG@ff{#1}}}}}}}
\def\PYG#1#2{\PYG@reset\PYG@toks#1+\relax+\PYG@do{#2}}

\expandafter\def\csname PYG@tok@cm\endcsname{\let\PYG@it=\textit\def\PYG@tc##1{\textcolor[rgb]{0.25,0.50,0.56}{##1}}}
\expandafter\def\csname PYG@tok@gs\endcsname{\let\PYG@bf=\textbf}
\expandafter\def\csname PYG@tok@go\endcsname{\def\PYG@tc##1{\textcolor[rgb]{0.20,0.20,0.20}{##1}}}
\expandafter\def\csname PYG@tok@cpf\endcsname{\let\PYG@it=\textit\def\PYG@tc##1{\textcolor[rgb]{0.25,0.50,0.56}{##1}}}
\expandafter\def\csname PYG@tok@kd\endcsname{\let\PYG@bf=\textbf\def\PYG@tc##1{\textcolor[rgb]{0.00,0.44,0.13}{##1}}}
\expandafter\def\csname PYG@tok@sc\endcsname{\def\PYG@tc##1{\textcolor[rgb]{0.25,0.44,0.63}{##1}}}
\expandafter\def\csname PYG@tok@ne\endcsname{\def\PYG@tc##1{\textcolor[rgb]{0.00,0.44,0.13}{##1}}}
\expandafter\def\csname PYG@tok@s2\endcsname{\def\PYG@tc##1{\textcolor[rgb]{0.25,0.44,0.63}{##1}}}
\expandafter\def\csname PYG@tok@kr\endcsname{\let\PYG@bf=\textbf\def\PYG@tc##1{\textcolor[rgb]{0.00,0.44,0.13}{##1}}}
\expandafter\def\csname PYG@tok@sb\endcsname{\def\PYG@tc##1{\textcolor[rgb]{0.25,0.44,0.63}{##1}}}
\expandafter\def\csname PYG@tok@gr\endcsname{\def\PYG@tc##1{\textcolor[rgb]{1.00,0.00,0.00}{##1}}}
\expandafter\def\csname PYG@tok@c1\endcsname{\let\PYG@it=\textit\def\PYG@tc##1{\textcolor[rgb]{0.25,0.50,0.56}{##1}}}
\expandafter\def\csname PYG@tok@ow\endcsname{\let\PYG@bf=\textbf\def\PYG@tc##1{\textcolor[rgb]{0.00,0.44,0.13}{##1}}}
\expandafter\def\csname PYG@tok@k\endcsname{\let\PYG@bf=\textbf\def\PYG@tc##1{\textcolor[rgb]{0.00,0.44,0.13}{##1}}}
\expandafter\def\csname PYG@tok@sr\endcsname{\def\PYG@tc##1{\textcolor[rgb]{0.14,0.33,0.53}{##1}}}
\expandafter\def\csname PYG@tok@sx\endcsname{\def\PYG@tc##1{\textcolor[rgb]{0.78,0.36,0.04}{##1}}}
\expandafter\def\csname PYG@tok@ni\endcsname{\let\PYG@bf=\textbf\def\PYG@tc##1{\textcolor[rgb]{0.84,0.33,0.22}{##1}}}
\expandafter\def\csname PYG@tok@kc\endcsname{\let\PYG@bf=\textbf\def\PYG@tc##1{\textcolor[rgb]{0.00,0.44,0.13}{##1}}}
\expandafter\def\csname PYG@tok@na\endcsname{\def\PYG@tc##1{\textcolor[rgb]{0.25,0.44,0.63}{##1}}}
\expandafter\def\csname PYG@tok@nd\endcsname{\let\PYG@bf=\textbf\def\PYG@tc##1{\textcolor[rgb]{0.33,0.33,0.33}{##1}}}
\expandafter\def\csname PYG@tok@mi\endcsname{\def\PYG@tc##1{\textcolor[rgb]{0.13,0.50,0.31}{##1}}}
\expandafter\def\csname PYG@tok@nc\endcsname{\let\PYG@bf=\textbf\def\PYG@tc##1{\textcolor[rgb]{0.05,0.52,0.71}{##1}}}
\expandafter\def\csname PYG@tok@nb\endcsname{\def\PYG@tc##1{\textcolor[rgb]{0.00,0.44,0.13}{##1}}}
\expandafter\def\csname PYG@tok@nv\endcsname{\def\PYG@tc##1{\textcolor[rgb]{0.73,0.38,0.84}{##1}}}
\expandafter\def\csname PYG@tok@mh\endcsname{\def\PYG@tc##1{\textcolor[rgb]{0.13,0.50,0.31}{##1}}}
\expandafter\def\csname PYG@tok@sh\endcsname{\def\PYG@tc##1{\textcolor[rgb]{0.25,0.44,0.63}{##1}}}
\expandafter\def\csname PYG@tok@gi\endcsname{\def\PYG@tc##1{\textcolor[rgb]{0.00,0.63,0.00}{##1}}}
\expandafter\def\csname PYG@tok@mb\endcsname{\def\PYG@tc##1{\textcolor[rgb]{0.13,0.50,0.31}{##1}}}
\expandafter\def\csname PYG@tok@kp\endcsname{\def\PYG@tc##1{\textcolor[rgb]{0.00,0.44,0.13}{##1}}}
\expandafter\def\csname PYG@tok@kn\endcsname{\let\PYG@bf=\textbf\def\PYG@tc##1{\textcolor[rgb]{0.00,0.44,0.13}{##1}}}
\expandafter\def\csname PYG@tok@vg\endcsname{\def\PYG@tc##1{\textcolor[rgb]{0.73,0.38,0.84}{##1}}}
\expandafter\def\csname PYG@tok@gt\endcsname{\def\PYG@tc##1{\textcolor[rgb]{0.00,0.27,0.87}{##1}}}
\expandafter\def\csname PYG@tok@vc\endcsname{\def\PYG@tc##1{\textcolor[rgb]{0.73,0.38,0.84}{##1}}}
\expandafter\def\csname PYG@tok@no\endcsname{\def\PYG@tc##1{\textcolor[rgb]{0.38,0.68,0.84}{##1}}}
\expandafter\def\csname PYG@tok@cs\endcsname{\def\PYG@tc##1{\textcolor[rgb]{0.25,0.50,0.56}{##1}}\def\PYG@bc##1{\setlength{\fboxsep}{0pt}\colorbox[rgb]{1.00,0.94,0.94}{\strut ##1}}}
\expandafter\def\csname PYG@tok@ge\endcsname{\let\PYG@it=\textit}
\expandafter\def\csname PYG@tok@mo\endcsname{\def\PYG@tc##1{\textcolor[rgb]{0.13,0.50,0.31}{##1}}}
\expandafter\def\csname PYG@tok@bp\endcsname{\def\PYG@tc##1{\textcolor[rgb]{0.00,0.44,0.13}{##1}}}
\expandafter\def\csname PYG@tok@ch\endcsname{\let\PYG@it=\textit\def\PYG@tc##1{\textcolor[rgb]{0.25,0.50,0.56}{##1}}}
\expandafter\def\csname PYG@tok@nl\endcsname{\let\PYG@bf=\textbf\def\PYG@tc##1{\textcolor[rgb]{0.00,0.13,0.44}{##1}}}
\expandafter\def\csname PYG@tok@vi\endcsname{\def\PYG@tc##1{\textcolor[rgb]{0.73,0.38,0.84}{##1}}}
\expandafter\def\csname PYG@tok@se\endcsname{\let\PYG@bf=\textbf\def\PYG@tc##1{\textcolor[rgb]{0.25,0.44,0.63}{##1}}}
\expandafter\def\csname PYG@tok@ss\endcsname{\def\PYG@tc##1{\textcolor[rgb]{0.32,0.47,0.09}{##1}}}
\expandafter\def\csname PYG@tok@sd\endcsname{\let\PYG@it=\textit\def\PYG@tc##1{\textcolor[rgb]{0.25,0.44,0.63}{##1}}}
\expandafter\def\csname PYG@tok@kt\endcsname{\def\PYG@tc##1{\textcolor[rgb]{0.56,0.13,0.00}{##1}}}
\expandafter\def\csname PYG@tok@o\endcsname{\def\PYG@tc##1{\textcolor[rgb]{0.40,0.40,0.40}{##1}}}
\expandafter\def\csname PYG@tok@w\endcsname{\def\PYG@tc##1{\textcolor[rgb]{0.73,0.73,0.73}{##1}}}
\expandafter\def\csname PYG@tok@gd\endcsname{\def\PYG@tc##1{\textcolor[rgb]{0.63,0.00,0.00}{##1}}}
\expandafter\def\csname PYG@tok@gh\endcsname{\let\PYG@bf=\textbf\def\PYG@tc##1{\textcolor[rgb]{0.00,0.00,0.50}{##1}}}
\expandafter\def\csname PYG@tok@gp\endcsname{\let\PYG@bf=\textbf\def\PYG@tc##1{\textcolor[rgb]{0.78,0.36,0.04}{##1}}}
\expandafter\def\csname PYG@tok@nf\endcsname{\def\PYG@tc##1{\textcolor[rgb]{0.02,0.16,0.49}{##1}}}
\expandafter\def\csname PYG@tok@m\endcsname{\def\PYG@tc##1{\textcolor[rgb]{0.13,0.50,0.31}{##1}}}
\expandafter\def\csname PYG@tok@s\endcsname{\def\PYG@tc##1{\textcolor[rgb]{0.25,0.44,0.63}{##1}}}
\expandafter\def\csname PYG@tok@cp\endcsname{\def\PYG@tc##1{\textcolor[rgb]{0.00,0.44,0.13}{##1}}}
\expandafter\def\csname PYG@tok@err\endcsname{\def\PYG@bc##1{\setlength{\fboxsep}{0pt}\fcolorbox[rgb]{1.00,0.00,0.00}{1,1,1}{\strut ##1}}}
\expandafter\def\csname PYG@tok@nt\endcsname{\let\PYG@bf=\textbf\def\PYG@tc##1{\textcolor[rgb]{0.02,0.16,0.45}{##1}}}
\expandafter\def\csname PYG@tok@il\endcsname{\def\PYG@tc##1{\textcolor[rgb]{0.13,0.50,0.31}{##1}}}
\expandafter\def\csname PYG@tok@si\endcsname{\let\PYG@it=\textit\def\PYG@tc##1{\textcolor[rgb]{0.44,0.63,0.82}{##1}}}
\expandafter\def\csname PYG@tok@mf\endcsname{\def\PYG@tc##1{\textcolor[rgb]{0.13,0.50,0.31}{##1}}}
\expandafter\def\csname PYG@tok@s1\endcsname{\def\PYG@tc##1{\textcolor[rgb]{0.25,0.44,0.63}{##1}}}
\expandafter\def\csname PYG@tok@c\endcsname{\let\PYG@it=\textit\def\PYG@tc##1{\textcolor[rgb]{0.25,0.50,0.56}{##1}}}
\expandafter\def\csname PYG@tok@gu\endcsname{\let\PYG@bf=\textbf\def\PYG@tc##1{\textcolor[rgb]{0.50,0.00,0.50}{##1}}}
\expandafter\def\csname PYG@tok@nn\endcsname{\let\PYG@bf=\textbf\def\PYG@tc##1{\textcolor[rgb]{0.05,0.52,0.71}{##1}}}

\def\PYGZbs{\char`\\}
\def\PYGZus{\char`\_}
\def\PYGZob{\char`\{}
\def\PYGZcb{\char`\}}
\def\PYGZca{\char`\^}
\def\PYGZam{\char`\&}
\def\PYGZlt{\char`\<}
\def\PYGZgt{\char`\>}
\def\PYGZsh{\char`\#}
\def\PYGZpc{\char`\%}
\def\PYGZdl{\char`\$}
\def\PYGZhy{\char`\-}
\def\PYGZsq{\char`\'}
\def\PYGZdq{\char`\"}
\def\PYGZti{\char`\~}
% for compatibility with earlier versions
\def\PYGZat{@}
\def\PYGZlb{[}
\def\PYGZrb{]}
\makeatother

\renewcommand\PYGZsq{\textquotesingle}

\begin{document}

\maketitle
\tableofcontents
\phantomsection\label{index::doc}



\chapter{David Pierson Bradway, Ph.D.}
\label{resume:david-pierson-bradway-ph-d}\label{resume::doc}\label{resume:web-enhanced-resume-cv}
\begin{DUlineblock}{0em}
\item[] \href{mailto:david.bradway@duke.edu}{david.bradway@duke.edu}
\item[] Research Scientist
\item[] Biomedical Engineering
\item[] Duke University
\item[] Durham, NC 27708 USA
\end{DUlineblock}


\section{Objective}
\label{resume:objective}\begin{itemize}
\item {} 
Career in research, visualization, data acquisition, and signal
processing

\item {} 
Engineering, research and development role in academia or industry

\end{itemize}


\section{Work Experience}
\label{resume:work-experience}\begin{itemize}
\item {} 
\textbf{Duke University} (Durham, NC, USA)

Research Scientist, 2014 - present

\item {} 
\textbf{Technical University of Denmark (DTU)} (Kongens Lyngby, Denmark)

Postdoctoral Researcher, 2013 - 2014
\begin{itemize}
\item {} 
Developed OpenCL software for processing 3-D Doppler ultrasound
data on the GPU

\item {} 
Presented results in conference abstract, poster, and proceedings

\item {} 
Completed pre-clinical feasibility study of cardiac vector flow
imaging and preparing peer-reviewed article

\end{itemize}

\item {} 
\textbf{Duke University} (Durham, NC, USA)

Graduate Research and Teaching Assistant, 2005 - 2013
\begin{itemize}
\item {} 
PhD project using ultrasound to noninvasively measure the heart's
mechanical properties

\item {} 
Organized and conducted out pre-clinical trials at Duke University
Medical Center

\item {} 
Presented results in conferences, proceedings and co-authored
articles

\end{itemize}

\item {} 
\textbf{Siemens Healthcare} (Issaquah, WA, USA)

Graduate Student Research Intern, 2008
\begin{itemize}
\item {} 
Worked within a research team in a multinational corporation

\item {} 
Added multiple focal zone ARFI excitation to research mode of
Acuson S2000 ultrasound scanner

\item {} 
Learned \href{http://www-03.ibm.com/software/products/en/clearcase}{version
control}
and \href{http://www.visualstudio.com/}{IDE} tools

\end{itemize}

\end{itemize}


\section{Education}
\label{resume:education}\begin{itemize}
\item {} 
\textbf{Duke University} (Durham, NC, USA)

\href{http://bme.duke.edu/grad}{Ph.D. in Biomedical Engineering}, May
2013.

\item {} 
\textbf{The Ohio State University (OSU)} (Columbus, OH, USA)

\href{http://ece.osu.edu/futurestudents/undergrad}{B.S. in Electrical and Computer
Engineering}, June
2005.

\end{itemize}


\section{Teaching}
\label{resume:teaching}\begin{itemize}
\item {} 
Ultrasound module of Medical Physics 731 (2015, 2016)

\item {} 
Guest lecturer for Biomedical Engineering 590 (2016)

\item {} 
Instructor at ``International Summer School on Advanced Ultrasound
Imaging'', Technical University of Denmark (June 2015)

\end{itemize}


\section{Invention Disclosures and Patents}
\label{resume:invention-disclosures-and-patents}\begin{itemize}
\item {} 
Duke IDF \#4547: ``Low-Cost, Portable Ultrasound Shear Wave Device for
Characterization of Material Viscoelasticity''

\item {} 
Duke IDF \#540: ``Visual Feedback for Improved Triggered Acquisitions
for Ultrasound Imaging''

\end{itemize}


\section{Honors and Activities}
\label{resume:honors-and-activities}\begin{itemize}
\item {} 
\href{http://www.whitaker.org/grants/fellows-scholars}{Whitaker International Program
Scholar} (2013)

\item {} 
Student Travel Support IEEE International Ultrasonics
Symposium, Dresden, Germany (2012)

\item {} 
\href{http://www.nsfgrfp.org/}{National Science Foundation Graduate Research
Fellow} (2005-2008)

\item {} 
\href{https://goldwater.scholarsapply.org/}{Goldwater Research Scholar}
(2004-2005)

\item {} 
\href{http://www.montanadeluz.org/}{Organized engineering design and build trip to Honduran
orphanage} (2004)

\item {} 
\href{http://ecos.osu.edu/}{Founded engineering community service group at Ohio
State} (2003)

\end{itemize}


\section{Professional Activities}
\label{resume:professional-activities}\begin{itemize}
\item {} 
Early Career Professional Development in Medical Imaging Workshop,
SPIE Medical Imaging (2015)

\item {} 
Associate Editor for the journal Ultrasonic Imaging (2016-)

\item {} 
Reviewer for the journals Ultrasound in Medicine and Biology,
Ultrasonic Imaging, IEEE Transactions on Medical Imaging

\end{itemize}


\section{Submitted proposals}
\label{resume:submitted-proposals}\begin{itemize}
\item {} 
``Sensor-enabled Ultrasound Probes for Volumetric Image Acquisition
and Interpretation: Proof of Concept in Pediatric Appendicitis''. Duke
– Wallace H. Coulter Translational Research Grants Program

\item {} 
``Portable Ultrasound Device for Staging Liver Fibrosis''. Duke –
Wallace H. Coulter Translational Research Grants Program

\end{itemize}


\section{Skills}
\label{resume:skills}\begin{itemize}
\item {} 
Advanced signal and imaging processing programming: Matlab and Python

\item {} 
Working knowledge of many tools and languages:
\href{http://git-scm.com/}{Git}, C/C++, OpenCL/CUDA, MS Office, and
Markdown

\item {} 
Focused on problem solving, signal and image analysis, scientific
computing, and experimental design

\item {} 
Strong written and verbal communication, and data visualization
display skills

\item {} 
Successful writer of fellowships and scholarships

\item {} 
Personal projects in mobile and embedded systems:
\href{http://www.arduino.cc/}{Arduino}, \href{http://www.raspberrypi.org/}{Raspberry
Pi}

\end{itemize}


\subsection{Journal Articles}
\label{resume:journal-articles}
{[}1–12{]}


\subsection{Abstracts and Proceedings}
\label{resume:abstracts-and-proceedings}
{[}13–38{]}


\section{Publications}
\label{resume:publications}
1. Fahey BJ, Nelson RC, Bradway DP, Hsu SJ, Dumont DM, et al. (2008) In
vivo visualization of abdominal malignancies with acoustic radiation
force elastography. Physics in medicine and biology 53: 279–93.
doi:\href{http://dx.doi.org/10.1088/0031-9155/53/1/020}{10.1088/0031-9155/53/1/020}

2. Fahey BJ, Nelson RC, Hsu SJ, Bradway DP, Dumont DM, et al. (2008) In
vivo guidance and assessment of liver radio-frequency ablation with
acoustic radiation force elastography. Ultrasound in medicine \& biology
34: 1590–603.
doi:\href{http://dx.doi.org/10.1016/j.ultrasmedbio.2008.03.006}{10.1016/j.ultrasmedbio.2008.03.006}

3. Nightingale K, Palmeri M, Zhai L, Frinkley K, Wang M, et al. (KR)
Impulsive acoustic radiation force: imaging approaches and clinical
applications. The Journal of the Acoustical Society of America 123:
3792. doi:\href{http://dx.doi.org/10.1121/1.2935460}{10.1121/1.2935460}

4. Nightingale K, Palmeri M, Dahl J, Bradway D, Hsu S, et al. (2009)
Elasticity imaging with acoustic radiation force: Methods and clinical
applications. Japanese journal of medical ultrasonics 36: 116.

5. Wolf PD, Eyerly SA, Bradway DP, Dumont DM, Bahnson TD, et al. (2011)
Near real time evaluation of cardiac radiofrequency ablation lesions
with intracardiac echocardiography based acoustic radiation force
impulse imaging. The Journal of the Acoustical Society of America 129:
2438. doi:\href{http://dx.doi.org/10.1121/1.3587978}{10.1121/1.3587978}

6. Eyerly SA, Bahnson TD, Koontz JI, Bradway DP, Dumont DM, et al.
(2012) Intracardiac acoustic radiation force impulse imaging: A novel
imaging method for intraprocedural evaluation of radiofrequency ablation
lesions. Heart rhythm: the official journal of the Heart Rhythm Society
9: 1855–1862.
doi:\href{http://dx.doi.org/10.1016/j.hrthm.2012.07.003}{10.1016/j.hrthm.2012.07.003}

7. Hollender P, Bradway D, Wolf P, Goswami R, Trahey G (2013)
Intracardiac acoustic radiation force impulse (aRFI) and shear wave
imaging in pigs with focal infarctions. IEEE transactions on
ultrasonics, ferroelectrics, and frequency control 60: 1669–1682.
doi:\href{http://dx.doi.org/10.1109/TUFFC.2013.2749}{10.1109/TUFFC.2013.2749}

8. Patel V, Dahl JJ, Bradway DP, Doherty JR, Lee SY, et al. (2014)
Acoustic Radiation Force Impulse Imaging (ARFI) on an IVUS Circular
Array. Ultrasonic Imaging 36: 98–111.
doi:\href{http://dx.doi.org/10.1177/0161734613511595}{10.1177/0161734613511595}

9. Eyerly SA, Bahnson TD, Koontz JI, Bradway DP, Dumont DM, et al.
(2014) Contrast in Intracardiac Acoustic Radiation Force Impulse Images
of Radiofrequency Ablation Lesions. Ultrasonic Imaging 36: 133–148.
doi:\href{http://dx.doi.org/10.1177/0161734613519602}{10.1177/0161734613519602}

10. Jensen JA, Rasmussen MF, Pihl MJ, Holbek S, Villagómez HCA, et al.
(2016) Safety assessment of advanced imaging sequences i: Measurements.
IEEE transactions on ultrasonics, ferroelectrics, and frequency control
63: 110–119.
doi:\href{http://dx.doi.org/10.1109/TUFFC.2015.2502987}{10.1109/TUFFC.2015.2502987}

11. Bottenus N, Long W, Zhang H, Jakovljevic M, Bradway D, et al. (2016)
Feasibility of swept synthetic aperture ultrasound imaging.
doi:\href{http://dx.doi.org/10.1109/TMI.2016.2524992}{10.1109/TMI.2016.2524992}

12. Long W, Hyun D, Choudhury KR, Bradway D, McNally P, et al. (2018)
Clinical utility of fetal short-lag spatial coherence imaging.
Ultrasound in medicine \& biology.
doi:\href{http://dx.doi.org/10.1016/j.ultrasmedbio.2017.12.006}{10.1016/j.ultrasmedbio.2017.12.006}

13. Hsu SJ, Bradway DP, Fahey BJ, Trahey GE (2007) Transthoracic
Acoustic Radiation Force Impulse Imaging of the Cardiac Cycle. In:
Ultrasonic measurement and imaging of tissue elasticity.

14. Bradway DP, Hsu SJ, Fahey BJ, Dahl JJ, Nichols TC, et al. (2007)
6B-6 Transthoracic Cardiac Acoustic Radiation Force Impulse Imaging: A
Feasibility Study. IEEE. pp. 448–451.
doi:\href{http://dx.doi.org/10.1109/ULTSYM.2007.121}{10.1109/ULTSYM.2007.121}

15. Fahey BJ, Nelson RC, Hsu SJ, Bradway DP, Dumont DM, et al. (2007)
6B-4 In Vivo Acoustic Radiation Force Impulse Imaging of Abdominal
Lesions. In: 2007 iEEE ultrasonics symposium proceedings. IEEE. pp.
440–443.
doi:\href{http://dx.doi.org/10.1109/ULTSYM.2007.119}{10.1109/ULTSYM.2007.119}

16. Bradway DP, Fahey BJ, Nelson RC, Trahey GE (2009) ARFI imaging of
abdominal ablation and liver lesion biopsy. In: International symposium
on ultrasonic imaging and tissue characterization, 2009. Available:
\href{http://uitc-symposium.org/2009\_abstracts.pdf}{http://uitc-symposium.org/2009\_abstracts.pdf}.

17. Husarik D, Nelson RC, Bradway DP, Fahey BJ, Nightingale KR, et al.
(2009) First Clinical Experience with Sonographic Elastography of the
Liver Using Acoustic Radiation Force Impulse (ARFI) Imaging. In: RSNA.
Available: \href{http://rsna2009.rsna.org/search}{http://rsna2009.rsna.org/search}.

18. Nelson RC, Bradway DP, Fahey BJ, Trahey GE (2009) Future Application
of Ultrasound: Acoustic Radiation Force Impulse (ARFI) Imaging. In:
AIUM. Available:
\href{http://www.aium.org/loginRequired/membersOnly/proceedings/2009.pdf}{http://www.aium.org/loginRequired/membersOnly/proceedings/2009.pdf}.

19. Bradway DP, Fahey BJ, Nelson RC, Trahey GE (2009) Recent Clinical
Results of Acoustic Radiation Force Impulse Imaging of Abdominal
Ablation. In: International tissue elasticity conference. Available:
\href{http://www.elasticityconference.org/prior\_conf/2009/PDF/2009Proceedings.pdf}{http://www.elasticityconference.org/prior\_conf/2009/PDF/2009Proceedings.pdf}.

20. Hsu SJ, Bradway DP, Bouchard RR, Hollender PJ, Wolf PD, et al.
(2010) Parametric pressure-volume analysis and acoustic radiation force
impulse imaging of left ventricular function. In: 2010 iEEE
international ultrasonics symposium. IEEE. pp. 698–701.
doi:\href{http://dx.doi.org/10.1109/ULTSYM.2010.5935661}{10.1109/ULTSYM.2010.5935661}

21. Hollender PJ, Bouchard RR, Hsu SJ, Bradway DP, Wolf PD, et al.
(2010) Intracardiac measurements of elasticity using Acoustic Radiation
Force Impulse (ARFI) methods: Temporal and spatial stability of shear
wave velocimetry. In: 2010 iEEE international ultrasonics symposium.
IEEE. pp. 698–701.
doi:\href{http://dx.doi.org/10.1109/ULTSYM.2010.5935946}{10.1109/ULTSYM.2010.5935946}

22. Bradway DP, Hsu SJ, Wolf PD, Trahey GE (2010) Acoustic Radiation
Force Impulse Imaging of Acute Myocardial Ischemia and Infarct. In:
International symposium on ultrasonic imaging and tissue
characterization. Available:
\href{http://uitc-symposium.org/2010\_abstracts.pdf}{http://uitc-symposium.org/2010\_abstracts.pdf}.

23. Bradway DP, Hsu SJ, Wolf PD, Trahey GE (2010) Transthoracic Acoustic
Radiation Force Impulse Imaging of Cardiac Function. In: International
tissue elasticity conference. Available:
\href{http://www.elasticityconference.org/prior\_conf/2010/PDF/2010Proceedings.pdf}{http://www.elasticityconference.org/prior\_conf/2010/PDF/2010Proceedings.pdf}.

24. Bradway DP, Rosenzweig SR, Doherty JR, Hyun D, Trahey GE (2011)
Recent Results and Advances in Transthoracic Cardiac Acoustic Radiation
Force Impulse Imaging. In: International symposium on ultrasonic imaging
and tissue characterization. Available:
\href{http://www.elasticityconference.org/prior\_conf/2011/PDF/2011ITECProceedings.pdf}{http://www.elasticityconference.org/prior\_conf/2011/PDF/2011ITECProceedings.pdf}.

25. Byram BC, Gianantonio DM, Bradway DP, Hyun D, Jakovljevic M, et al.
(2011) Direct in vivo Myocardial Infarct Visualization Using 3D
Ultrasound and Passive Strain Contrast. In: International tissue
elasticity conference. Available:
\href{http://www.elasticityconference.org/prior\_conf/2011/PDF/2011ITECProceedings.pdf}{http://www.elasticityconference.org/prior\_conf/2011/PDF/2011ITECProceedings.pdf}.

26. Byram BC, Bradway DP, Jakovljevic M, Gianantonio D, Hyun D, et al.
(2011) Direct In Vivo Myocardial Infarct Visualization Using 3D
Ultrasound and Passive Strain Contrast. In: IEEE ultrasonics symp.
doi:\href{http://dx.doi.org/10.1109/ULTSYM.2011.0007}{10.1109/ULTSYM.2011.0007}

27. Bradway DP, Hollender PJ, Goswami R, Wolf PD, Trahey GE (2012)
Feasibility and safety of transthoracic cardiac acoustic radiation force
impulse imaging methods. In: 2012 iEEE international ultrasonics
symposium. IEEE. pp. 2027–2030.
doi:\href{http://dx.doi.org/10.1109/ULTSYM.2012.0507}{10.1109/ULTSYM.2012.0507}

28. Bradway DP, Hollender PJ, Goswami R, Wolf PD, Trahey GE (2012)
Transthoracic Cardiac Acoustic Radiation Force Impulse Imaging: in vivo
Feasibility, Methods, and Initial Results. In: International symposium
on ultrasonic imaging and tissue characterization, 2012. Available:
\href{http://uitc-symposium.org/2012\_abstracts.pdf}{http://uitc-symposium.org/2012\_abstracts.pdf}.

29. Hollender PJ, Bradway DP, Goswami R, Wolf PD, Trahey GE (2012)
Acoustic radiation force techniques for imaging cardiac infarct in vivo:
methods and initial results. In: International symposium on ultrasonic
imaging and tissue characterization. Available:
\href{http://uitc-symposium.org/2012\_abstracts.pdf}{http://uitc-symposium.org/2012\_abstracts.pdf}.

30. Eyerly SA, Bahnson T, Koontz J, Bradway DP, Dumont D, et al. (2012)
Confirmation of Cardiac Radiofrequency Ablation Treatment Using
Intra-Procedure Acoustic Radiation Force Impulse Imaging. In: IEEE
ultrasonics symposium.
doi:\href{http://dx.doi.org/10.1109/ULTSYM.2012.0509}{10.1109/ULTSYM.2012.0509}

31. Hollender PJ, Bradway DP, Wolf PD, Goswami R, Trahey GE (2012)
Intracardiac ARF-driven Shear Wave Velocimetry to Estimate Regional
Myocardial Stiffness and Contractility in Pigs with Focal Infarctions.
In: IEEE ultrasonics symposium.
doi:\href{http://dx.doi.org/10.1109/ULTSYM.2012.0508}{10.1109/ULTSYM.2012.0508}

32. Goswami R, Bradway D, Kisslo J, Trahey G (2013) Novel Application of
Acoustic Radiation Force Impulse Imaging in Transthoracic
Echocardiography. In: Journal of the american college of cardiology.
American College of Cardiology Foundation, Vol. 61. p. E1090.
doi:\href{http://dx.doi.org/10.1016/S0735-1097(13)61090-6}{10.1016/S0735-1097(13)61090-6}

33. Patel V, Dahl JJ, Bradway DP, Doherty JR, Smith SW (2013) Acoustic
radiation force impulse imaging on an IVUS circular array. In: 2013 iEEE
international ultrasonics symposium (iUS). IEEE. pp. 773–776.
doi:\href{http://dx.doi.org/10.1109/ULTSYM.2013.0199}{10.1109/ULTSYM.2013.0199}

34. Bradway DP, Pihl MJ, Krebs andreas, Tomov BG, Kjær CS, et al. (2014)
Real-time GPU implementation of transverse oscillation vector velocity
flow imaging. In: SPIE medical imaging.Vol. 9040. pp. 90401Y–90401Y–6.
doi:\href{http://dx.doi.org/10.1117/12.2043582}{10.1117/12.2043582}

35. Bradway DP, Hansen KL, Nielsen MB, Jensen JA (2015) Transverse
oscillation vector flow imaging for transthoracic echocardiography. In:
SPIE medical imaging. pp. 941902–941902–7.
doi:\href{http://dx.doi.org/10.1117/12.2081145}{10.1117/12.2081145}

36. Hollender P, Bottenus N, Bradway D, Trahey G (2017) Single track
location comb-push ultrasound shear elastography (sTL-cUSE). In:
Ultrasonics symposium (iUS), 2017 iEEE international. IEEE. pp. 1–4.
doi:\href{http://dx.doi.org/10.1109/ULTSYM.2017.8091901}{10.1109/ULTSYM.2017.8091901}

37. Kakkad V, Ferlauto H, Bradway D, Heyde B, Kisslo J, et al. (2017)
Clinical feasibility of a noninvasive method to interrogate myocardial
function via strain and acoustic radiation force-derived stiffness. In:
Ultrasonics symposium (iUS), 2017 iEEE international. IEEE. pp. 1–1.
doi:\href{http://dx.doi.org/10.1109/ULTSYM.2017.8092067}{10.1109/ULTSYM.2017.8092067}

38. Long W, Hyun D, Choudhury K, Bradway D, McNally P, et al. (2017)
Translation of fetal short-lag spatial coherence (sLSC) imaging into
clinical practice: A pilot study. In: Ultrasonics symposium (iUS), 2017
iEEE international. IEEE. pp. 1–1.
doi:\href{http://dx.doi.org/10.1109/ULTSYM.2017.8091968}{10.1109/ULTSYM.2017.8091968}



\renewcommand{\indexname}{Index}
\printindex
\end{document}
